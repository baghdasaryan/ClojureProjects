% Created 2013-08-17 Sat 21:07
\documentclass[11pt]{article}
\usepackage[utf8]{inputenc}
\usepackage[T1]{fontenc}
\usepackage{fixltx2e}
\usepackage{graphicx}
\usepackage{longtable}
\usepackage{float}
\usepackage{wrapfig}
\usepackage[normalem]{ulem}
\usepackage{textcomp}
\usepackage{marvosym}
\usepackage{wasysym}
\usepackage{latexsym}
\usepackage{amssymb}
\usepackage{amstext}
\usepackage{hyperref}
\tolerance=1000
\author{Brian Beckman}
\date{\today}
\title{Experiments in Literate Programming}
\hypersetup{
  pdfkeywords={},
  pdfsubject={},
  pdfcreator={Emacs 24.2.1 (Org mode 8.0.7)}}
\begin{document}

\maketitle
\tableofcontents

\begin{abstract}
Fast, parallel factoring of integers is the ``Hello, world!''
of hackers and of good guys.

The problem looks like this: given a big integer $n\in\mathbb{N}$ such
 that \[n = p q\] where $p$ and $q$
are big primes, find $p$ and $q$.  RSA uses numbers like $n$ as public encryption
keys. Anyone can encrypt you a message using $n$.
Decrypting is easy only if you know $p$ and $q$.

Because RSA and most of 
internet security depends on the assumption that factoring is
hard, this problem is of critical interest. Whoever can
`break' RSA by factoring keys will `own the world' for a short time,
until internet security is reformulated in some new way. The biggest
key factored so far is
\end{abstract}

\begin{figure}[H]
\label{biggest-factored-rsa-key}
\begin{verbatim}
(defn wrapped-lines-to-bigint [& strs]
  (bigint (apply str strs)))

(def RSA-768
  (wrapped-lines-to-bigint
   "12301866845301177551304949583849627207728535695953347921"
   "97322452151726400507263657518745202199786469389956474942"
   "77406384592519255732630345373154826850791702612214291346"
   "16704292143116022212404792747377940806653514195974598569"
   "02143413"
   ))

(def TEST
  (*
   (wrapped-lines-to-bigint
    "33478071698956898786044169848212690817704794983713768568"
    "91243138898288379387800228761471165253174308773781446799"
    "9489")

   (wrapped-lines-to-bigint
    "36746043666799590428244633799627952632279158164343087642"
    "67603228381573966651127923337341714339681027009279873630"
    "8917")))

(== RSA-768 TEST)
\end{verbatim}
\end{figure}
$\longrightarrow$
\begin{verbatim}
true
\end{verbatim}

\section{Literate Programming With Org-Mode and Clojure}
\label{sec-1}

Today's \label{internet-security}internet security depends upon factoring being hard. If
factoring isn't hard, \hyperref[internet-security]{internet security} must be rethought in new
terms.

First, add the following to the \verb~:dependencies~ section of your \href{https://github.com/technomancy/leiningen}{Leiningen}
\emph{project.clj} file:

\begin{verbatim}
1  [org.clojure/core.contracts "0.0.5"]
\end{verbatim}
\section{First Experiment}
\label{sec-2}
In line 1 of section \ref{sec-1}, \hyperref[sec-1]{the Introduction}, and
perhaps even in \ref{sec:more}, \hyperref[sec:more]{But Wait, There's More}, the version of
\textbf{contracts} was specified; this doesn't yet have anything to do with
\hyperref[internet-security]{internet security}.

Next, shift attention to the file ``core.clj,'' which implements the
principal functions of our demonstration.

\begin{verbatim}
2  (ns big-prime.core
3  (:import java.util.Random)
4  (:use [big-prime.utils]
5        [big-prime.sqrt :as nt]
6        [clojure.core.contracts :as contracts]
7        [clojure.set :only [difference]]
8        ))
\end{verbatim}
\section{One More}
\label{sec-3}
\label{sec:more} 

The mass of the sun is M$_{\text{sun}~}$=~1.989~\texttimes{}~10$^{\text{30}}$~kg.  The radius of
the sun is R$_{\text{sun}}$~=~6.96~\texttimes{}~10$^{\text{8}}$~m.
% Emacs 24.2.1 (Org mode 8.0.7)
\end{document}