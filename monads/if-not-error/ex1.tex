% Created 2013-09-07 Sat 07:05
\documentclass[11pt]{article}
\usepackage[utf8]{inputenc}
\usepackage[T1]{fontenc}
\usepackage{fixltx2e}
\usepackage{graphicx}
\usepackage{longtable}
\usepackage{float}
\usepackage{wrapfig}
\usepackage[normalem]{ulem}
\usepackage{textcomp}
\usepackage{marvosym}
\usepackage{wasysym}
\usepackage{latexsym}
\usepackage{amssymb}
\usepackage{amstext}
\usepackage{hyperref}
\tolerance=1000
\usepackage{tikz}
\usepackage{tikz}
\usepackage{tikz-cd}
\usetikzlibrary{matrix,arrows,positioning,scopes,chains}
\tikzset{node distance=2cm, auto}
\author{Brian Beckman}
\date{\today}
\title{Fluent, Composable Error Handling}
\hypersetup{
  pdfkeywords={},
  pdfsubject={},
  pdfcreator={Emacs 24.3.1 (Org mode 8.0.7)}}
\begin{document}

\maketitle
\tableofcontents


\section{Introduction}
\label{sec-1}

Consider a program composed of a sequence of \emph{serially dependent}
computations, any of which produces either a value to feed to the
next computation or an error value. If any component in the sequence
produces an error value, that value should be the result of the
entire sequence and all computations following the erroring
computation should be abandoned.

The desired behavior is similar to that of the Maybe
monad,\footnote{\url{http://en.wikipedia.org/wiki/Monad_(functional_programming)}\#The$_{\text{Maybe}}$$_{\text{monad}}$}
the difference being that \emph{Maybe} just produce \emph{Nothing} if anything
goes wrong. The consumer of the computation doesn't know what stage
of the pipeline failed nor any details at all about the error.
\emph{Maybe} suppresses all that. Such a situation is not tolerable in
the real world. Consider the example of a database retrieval
followed by a few web-service calls followed by a filter and
transformation followed by a logging call followed by output to UI
components. If something goes wrong in this sequence of
computations, we need to know exactly where and as much detail as
we can get about the failure. But we certainly don't want any
computations downstream of the failure to be attempted.


\subsection{How-To's}
\label{sec-1-1}
This is a literate program.\footnote{See
  \url{http://en.wikipedia.org/wiki/Literate_programming}.} That means that
source code \emph{and} documentation spring from the same, plain-text
source files. That gives us a fighting chance of keeping knowledge
and source coherent.

This file is named \emph{ex1.org}. It's an outline in plain text with
rather obvious structure. Top-level headlines get a single star;
second-level headlines get two stars, and so on; \LaTeX{} can be
freely written anywhere; source-code examples abound to
copy-and-paste, and text explaining how to build and run the source
is nearby.

You can edit the file with any plain-text editor. Emacs offers some
automation in generating the typeset output, \emph{ed1.pdf}, and the
source code of the application right out of the \emph{org} file. To
generate source code, issue the emacs command
\verb|org-babel-tangle|. To generate documentation, issue the emacs
command \verb|org-latex-export-to-pdf|.

If you don't want to use emacs, by all means use \emph{any} plain-text
editor. We are working on a batch process via \emph{make} so that you can
just clone the repo, make whatever edits you like, type \emph{make}, and
have a complete PDF file and a complete directory full of source
code.
\section{Tangle to Leiningen}
\label{sec-2}

\subsection{Files in the Project Directory}
\label{sec-2-1}
In our example, the top-level directory doesn't have a name -- 
put our \emph{org} file in that directory. The Leiningen project directory
will have the same name as our \emph{org} file. Our \emph{org} file is named
\verb+ex1.org+ and we want a directory tree rooted at \verb+ex1+
exactly as above.

Start with the contents of the project directory, \verb+ex1+. Each
org-mode babel source-code block will name a file path -- including
sub-directories -- after a \verb+:tangle+ keyword on the
\texttt{\#+BEGIN\_SRC} command of org-mode.
\subsubsection{.Gitignore}
\label{sec-2-1-1}
First, we must create the \verb+.gitignore+ file that tells
\verb+git+ not to check in the ephemeral \emph{ejecta} of build
processes like \verb+maven+ and \verb+javac+. When we gain more
confidence and adoption with tangle and \LaTeX{}, we will even
ignore the PDF file and the generated source tree, saving \emph{only}
the \emph{org} file in the repository.

\subsubsection{README.md}
\label{sec-2-1-2}
Next, we produce a \verb+README.md+ in \verb+markdown+ syntax for
the entire project:
\begin{verbatim}
# ex1
A Clojure library designed to do SOMETHING. 
## Usage
TODO
## License
Copyright © 2013 TODO
\end{verbatim}

\subsubsection{project.clj}
\label{sec-2-1-3}
Next is the \verb+project.clj+ file required by Leiningen for fetching
dependencies, loading libraries, and other housekeeping. If you are
running the Clojure REPL inside emacs, you must visit this file \emph{after
tangling it out of the org file}, and then run
\begin{verbatim}
M-x nrepl-jack-in
\end{verbatim}
in that buffer (see more in section
\ref{sec:emacs-repl}). 
\begin{figure}[H]
\label{project-file}
\begin{verbatim}
(defproject ex1 "0.1.0-SNAPSHOT"
  :description "Project Fortune's Excel Processor"
  :url "http://example.com/TODO"
  :license {:name "TODO"
            :url "TODO"}
  :dependencies [[org.clojure/clojure  "1.5.1"]
                 [org.clojure/data.zip "0.1.1"]
                 [dk.ative/docjure     "1.6.0"]
                ])
\end{verbatim}
\end{figure}
\subsection{The Documentation Subdirectory}
\label{sec-2-2}
Mimicking Leiningen's documentation subdirectory, it contains the
single file \verb+intro.md+, again in \verb+markdown+ syntax.
\begin{verbatim}
# Introduction to ex1
TODO: The project documentation is the .org file that produced 
this output, but it still pays to read
http://jacobian.org/writing/great-documentation/what-to-write/
\end{verbatim}

\subsection{Core Source File}
\label{sec-2-3}
By convention, the core source files go in a subdirectory named
\verb+./ex1/src/ex1+. This convention allows the Clojure namespaces
to map to Java packages.

The following is our core source file, explained in small pieces.
The \emph{org} file contains a spec for emitting the tangled source at
this point. This spec is not visible in the generated PDF file,
because we want to individually document the small pieces. The
invisible spec simply gathers up the source of the small pieces from
out of their explanations and then emits them into the source
directory tree, using another tool called
\emph{noweb}.\footnote{\url{http://orgmode.org/manual/Noweb-reference-syntax.html}}
This is not more complexity for you to learn, rather it is just a
way for you to feel comfortable with literate-programming magic.
\subsubsection{The Namespace}
\label{sec-2-3-1}
First, we must mention the libraries we're using. This is pure
ceremony, and we get to the meat of the code immediately after. These
library-mentions correspond to the \verb|:dependencies| in the
\verb|project.clj| file above. Each \verb|:use| or \verb|:require|
below must correspond to either an explicit dependency in the
\verb|project.clj| file or to one of several implicitly loaded
libraries. Leiningen loads libraries by processing the
\verb|project.clj| file above. We bring symbols from those libraries
into our namespace so we can use the libraries in our core routines.

To ingest and compile raw Excel spreadsheets, we use the built-in
libraries \verb|clojure.zip| for tree navigation and
\verb|clojure.xml| for XML parsing, plus the third-party libraries
\verb|clojure.data.zip.xml| and \verb|dk.ative.docjure.spreadsheet|.
The following brings these libraries into our namespace:
\begin{figure}[H]
\label{main-namespace}
\begin{verbatim}
(ns ex1.core
  (:use [clojure.data.zip.xml :only (attr text xml->)]
        [dk.ative.docjure.spreadsheet] ) 
  (:require [clojure.xml :as xml]
            [clojure.zip :as zip]))
\end{verbatim}
\end{figure}
\subsubsection{Data Instances}
\label{sec-2-3-2}
Next, we create a couple of data instances to manipulate later in our
unit tests. The first one ingests a trivial XML file and the second
one converts the in-memory data structure into a
\emph{zipper},\footnote{\url{http://richhickey.github.io/clojure/clojure.zip-api.html}}
a very modern, functional tree-navigation facility. These instances
will test our ability to freely navigate the raw XML form of Excel
spreadsheets:
\begin{figure}[H]
\label{main-zippered}
\begin{verbatim}
(def xml (xml/parse "myfile.xml"))
(def zippered (zip/xml-zip xml))
\end{verbatim}
\end{figure}
\subsubsection{A Test Excel Spreadsheet}
\label{sec-2-3-3}
Finally, we use \verb|docjure| to emit a test Excel spreadsheet, which
we will read in our unit tests and verify some operations on it. This
code creates a workbook with a single sheet in a rather obvious way,
picks out the sheet and its header row, and sets some visual
properties on the header row. We can open the resulting spreadsheet in
Excel after running \verb|lein test| and verify that the
\verb|docjure| library works as advertised.
\begin{figure}[H]
\label{docjure-test-spreadsheet}
\begin{verbatim}
(let [wb (create-workbook "Price List"
                          [["Name"       "Price"]
                           ["Foo Widget" 100]
                           ["Bar Widget" 200]])
      sheet (select-sheet "Price List" wb)
      header-row (first (row-seq sheet))]
  (do
    (set-row-style!
      header-row
      (create-cell-style! wb
        {:background :yellow,
         :font       {:bold true}}))
    (save-workbook! "spreadsheet.xlsx" wb)))
\end{verbatim}
\end{figure}
\subsection{Core Unit-Test File}
\label{sec-2-4}
Unit-testing files go in a subdirectory named \verb+./ex1/test/ex1+.
Again, the directory-naming convention enables valuable shortcuts
from Leiningen.

As with the core source files, we include the built-in and downloaded
libraries, but also the \verb|test framework| and the \verb|core|
namespace, itself, so we can test the functions in the core.
\begin{figure}[H]
\label{main-test-namespace}
\begin{verbatim}
(ns ex1.core-test
  (:use [clojure.data.zip.xml :only (attr text xml->)]
        [dk.ative.docjure.spreadsheet]
  )
  (:require [clojure.xml :as xml]
            [clojure.zip :as zip]
            [clojure.test :refer :all]
            [ex1.core :refer :all]))
\end{verbatim}
\end{figure}

We now test that the zippered XML file can be accessed by the \emph{zipper}
operators. The main operator of interest is \verb|xml->|, which acts
a lot like Clojure's
\emph{fluent-style} \footnote{\url{http://en.wikipedia.org/wiki/Fluent_interface}}
\emph{threading} operator
\verb|->|.\footnote{\url{http://clojuredocs.org/clojure_core/clojure.core/-\%3E}}
It takes its first argument, a zippered XML file in this case, and
then a sequence of functions to apply. For instance, the following
XML file, when subjected to the functions \verb|:track|,
\verb|:name|, and \verb|text|, should produce \verb|'("Track one" "Track two")|
\begin{verbatim}
<songs>
  <track id="t1"><name>Track one</name></track>
  <ignore>pugh!</ignore>
  <track id="t2"><name>Track two</name></track>
</songs>
\end{verbatim}
Likewise, we can dig into the attributes with natural accessor
functions \footnote{Clojure treats colon-prefixed keywords as functions that
fetch the corresponding values from hashmaps, rather like the dot
operator in Java or JavaScript; Clojure also treats hashmaps as
functions of their keywords: the result of the function call
$\texttt{(\{:a 1\} :a)}$ is the same as the result of the function call
$\texttt{(:a \{:a 1\})}$}\#+name: docjure-test-namespace

\begin{figure}[H]
\label{test-zippered}
\begin{verbatim}
(deftest xml-zipper-test
  (testing "xml and zip on a trivial file."
    (are [a b] (= a b)
      (xml-> zippered :track :name text) '("Track one" "Track two")
      (xml-> zippered :track (attr :id)) '("t1" "t2"))))
\end{verbatim}
\end{figure}

Next, we ensure that we can faithfully read back the workbook we
created \emph{via} \verb|docjure|. Here, we use Clojure's
\verb|thread-last| macro to achieve fluent style:
\begin{figure}[H]
\label{test-docjure-read}
\begin{verbatim}
(deftest docjure-test
  (testing "docjure read"
    (is (= 

      (->> (load-workbook "spreadsheet.xlsx")
           (select-sheet "Price List")
           (select-columns {:A :name, :B :price}))

      [{:name "Name"      , :price "Price"}, ; don't forget header row
       {:name "Foo Widget", :price 100.0  },
       {:name "Bar Widget", :price 200.0  }]

      ))))
\end{verbatim}
\end{figure}
\section{A REPL-based Solution}
\label{sec-3}
\label{sec:emacs-repl}
To run the REPL for interactive programming and testing in org-mode,
take the following steps:
\begin{enumerate}
\item Set up emacs and nRepl (TODO: explain; automate)
\item Edit your init.el file as follows (TODO: details)
\item Start nRepl while visiting the actual |project-clj| file.
\item Run code in the org-mode buffer with \verb|C-c C-c|; results of
evaluation are placed right in the buffer for inspection; they are
not copied out to the PDF file.
\end{enumerate}

\begin{verbatim}
[(xml-> zippered :track :name text)        ; ("Track one" "Track two")
 (xml-> zippered :track (attr :id))]       ; ("t1" "t2")
\end{verbatim}

\begin{verbatim}
(->> (load-workbook "spreadsheet.xlsx")
     (select-sheet "Price List")
     (select-columns {:A :name, :B :price}))
\end{verbatim}

\begin{verbatim}
(run-all-tests)
\end{verbatim}
\section{References}
\label{sec-4}

\section{Conclusion}
\label{sec-5}
Fu is Fortune.
% Emacs 24.3.1 (Org mode 8.0.7)
\end{document}
